\documentclass[../main.tex]{subfiles}

\begin{document}

\section{BQP subroutine}

\subsection{Main result}

\begin{theorem}
$\mathrm{BQP}^\mathrm{BQP} = \mathrm{BQP}$ \cite{BBBV97}
\end{theorem}
\begin{proof}
\begin{claim}
$\mathrm{BQP}\subseteq\mathrm{BQP}^\mathrm{BQP}$
\end{claim}
\begin{proof}
Trivially.
\end{proof}
\begin{claim}
$\mathrm{BQP}^\mathrm{BQP}\subseteq\mathrm{BQP}$
\end{claim}
\begin{proof}

To show that circuit for language in $\mathrm{BQP}^\mathrm{BQP}$ is in $\mathrm{BQP}$, we need to show that the circuit is poly size, the circuit gives correct output with constant probability $>\frac{1}{2}$ and the superposition outputted by the $\mathrm{BQP}$ oracle will not affect the commputation.

\noindent In worst case, poly-many calls are made to the oracle. SInce any $\mathrm{BQP}$ ckt is of poly-size, any ckt in $\mathrm{BQP}^\mathrm{BQP}$ can have size which is atmost product of 2 polynomials (which is still poly size).

\noindent Success probability of any $\mathrm{BQP}$ algorithm that runs in time $\mathrm{T}(n)$ can be boosted to $1-\epsilon$ success probability that runs in time $\mathrm{T}(n)\log(\frac{1}{\epsilon})$. Using this, we can ensure that success probability of any ckt in $\mathrm{BQP}^\mathrm{BQP}$ is some constant $>\frac{1}{2}$.

\noindent We only need the output state from the $\mathrm{BQP}$ oracle. This is easy to obtain since we can reversibly uncompute and end up with only the input and output states.

\noindent Using the above three points, we can see that $\mathrm{BQP}^\mathrm{BQP}\subseteq\mathrm{BQP}$.

\end{proof}
\noindent Using above two claims, we can easily see that $\mathrm{BQP}^\mathrm{BQP}=\mathrm{BQP}$.
\end{proof}

\end{document}
