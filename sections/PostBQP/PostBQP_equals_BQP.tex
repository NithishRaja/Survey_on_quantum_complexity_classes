\documentclass[../main.tex]{subfiles}

\begin{document}

\subsection{Prerequsites}

\subsubsection{Postselection}

Postselect means to condition the probability space upon the occurrence of an event.

\noindent In computations that involve randomness, to postselect on an event means to disregard all computational paths where the event does not occur.

\subsubsection{PostBQP}

$\mathrm{PostBQP}$ is the class of problems that can be solved by a $\mathrm{BQP}$ circuit that can postselect.

Formally, every language $\mathrm{L}$ in $\mathrm{PostBQP}$ has a family of circuits $\{C_n\}$ s.t.
\begin{enumerate}
	\item After applying $\{C_n\}$ to $\ket{00\hdots 0}\otimes\ket{x}$, the first qubit had a non-zero probability of being $\ket{1}$
	\item If $x \in \mathrm{L}$, $\mathrm{Pr}[\text{second qubit = } \ket{1} | \text{first qubit = } \ket{1}] \ge \frac{2}{3}$
	\item If $x \not\in \mathrm{L}$, $\mathrm{Pr}[\text{second qubit = } \ket{1} | \text{first qubit = } \ket{1}] \le \frac{1}{3}$
\end{enumerate}

\noindent Success probability of $\mathrm{PostBQP}$ machines can be boosted in the same way the success probability of a $\mathrm{BQP}$ machine is boosted.

\subsection{Main result}

\begin{theorem}
	$\mathrm{PostBQP} = \mathrm{PP}$ \cite{Aar05}
\end{theorem}
\begin{proof}
	\begin{claim}
		$\mathrm{PostBQP} \subseteq \mathrm{PP}$
	\end{claim}
	\begin{proof}
		W.l.o.g we assume that the circuit only contains hadamard and toffoli gates. Next we build the computation tree. Now, check if $\ket{11}$ occurs more than $\ket{10}$. This is simply computing majority which is in $\mathrm{PP}$. Clearly, this is a polytime reduction from any problem in $\mathrm{PostBQP}$ to majority.
	\end{proof}
	\begin{claim}
		$\mathrm{PP} \subseteq \mathrm{PostBQP}$
	\end{claim}
	\begin{proof}
		One way to show $\mathrm{PP}\subseteq\mathrm{PostBQP}$ is to give a $\mathrm{PostBQP}$ algorithm to calculate majority (calculating majority is $\mathrm{PP-complete}$).\\
		\noindent Consider function $f:\{0,1\}^n\rightarrow\{0,1\}$ s.t. $s = |\{x|f(x)=1\}|$. Our $\mathrm{PostBQP}$ algorithm needs to differentiate between the cases $s<2^{n/2}$ and $s\ge 2^{n/2}$.
		\begin{center}
		\label{qckt:PP_inside_PostBQP}
		\begin{quantikz}
			\lstick{$\ket{0}^{\otimes n}$} & \qwbundle{n} & \gate{H^{\otimes n}} & \gate[wires=2][2cm]{U_f} & \gate{H^{\otimes n}} & \qw \\
			& \lstick{$\ket{0}$} & \qw & \qw & \qw & \\
		\end{quantikz}
		\end{center}
		\begin{equation*}
			\ket{0^{\otimes n}}\ket{0} \xrightarrow{H^{\otimes n}\otimes I} \frac{1}{\sqrt{2^n}}\sum_{x\in\{0,1\}^n}\ket{x}\ket{f(x)}
		\end{equation*}
		\noindent Applying hadamard on any $\ket{x}$ gives us an equal superposition of all states with amplitudes being $\pm\frac{1}{\sqrt{2^n}}$.\\
		\noindent We apply $H^{\otimes n}$ on first register and look only at $\ket{00\hdots 0}$. Each state $\ket{x}$ contributes $\frac{1}{\sqrt{n}}$ to amplitude of $\ket{00\hdots 0}$ upon applying $H$ to the first register
		\begin{equation*}
			\frac{1}{\sqrt{2^n}}\sum_{x\in\{0,1\}^n}\ket{x}\ket{f(x)} \xrightarrow{H^{\otimes n}\otimes I} \ket{00\hdots 0}\sum_{x\in \{0,1\}^n}\ket{f(x)} + \Phi
		\end{equation*}
		\noindent Now, we postselect on the first register being $\ket{00\hdots 0}$. The second register will then become
		\begin{equation*}
			\ket{\psi} = \frac{(2^n-s)\ket{0}+s\ket{1}}{\sqrt{(2^n-s)^2+s^2}}\tag{$s=|\{x|f(x)=1\}|$}
		\end{equation*}
		\noindent Now prepare the state $\alpha\ket{0}\ket{\psi}+\beta\ket{1}H\ket{\psi}$ and postselect on the second register being $\ket{1}$. The first register becomes
		\begin{equation*}
			\frac{\alpha s\ket{0}+\beta\sqrt{1/2}(2^n-2s)\ket{1}}{\sqrt{\alpha^2s^2+(\beta^2/2)(2^n-2s)^2}}
		\end{equation*}
		\noindent By setting $\alpha$  and $\beta$ to appropriate values, we can distinguish between the cases $s<2^{n/2}$ and $s\ge 2^{n/2}$.
	\end{proof}
	\noindent Combining the above claims, we get $\mathrm{PostBQP} = \mathrm{PP}$.
\end{proof}

\end{document}
