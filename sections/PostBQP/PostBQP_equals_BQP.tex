\documentclass[../main.tex]{subfiles}

\begin{document}

\subsection{Prerequsites}

\subsubsection{Postselection}

Postselect means to condition the probability space upon the occurrence of an event.

\noindent In computations that involve randomness, to postselect on an event means to disregard all computational paths where the event does not occur.

\subsubsection{PostBQP}

$\mathrm{PostBQP}$ is the class of problems that can be solved by a $\mathrm{BQP}$ circuit that can postselect.

Formally, every language $\mathrm{L}$ in $\mathrm{PostBQP}$ has a family of circuits $\{C_n\}$ s.t.
\begin{enumerate}
	\item After applying $\{C_n\}$ to $\ket{00\hdots 0}\otimes\ket{x}$, the first qubit had a non-zero probability of being $\ket{1}$
	\item If $x \in \mathrm{L}$, $\mathrm{Pr}[\text{second qubit = } \ket{1} | \text{first qubit = } \ket{1}] \ge \frac{2}{3}$
	\item If $x \not\in \mathrm{L}$, $\mathrm{Pr}[\text{second qubit = } \ket{1} | \text{first qubit = } \ket{1}] \le \frac{1}{3}$
\end{enumerate}

\noindent Success probability of $\mathrm{PostBQP}$ machines can be boosted in the same way the success probability of a $\mathrm{BQP}$ machine is boosted.

\subsection{Main result}

\begin{theorem}
	$\mathrm{PostBQP} = \mathrm{PP}$
\end{theorem}
\begin{proof}
	\begin{claim}
		$\mathrm{PostBQP} \subset \mathrm{PP}$
	\end{claim}
	\begin{proof}
		proof of claim 1
	\end{proof}
	\begin{claim}
		$\mathrm{PP} \subset \mathrm{PostBQP}$
	\end{claim}
	\begin{proof}
		proof of claim 2
	\end{proof}
\end{proof}

\end{document}
